\begin{abstract}

		\section*{Kurzdarstellung}

		Die historische Entwicklung der Betriebssysteme bis hin zum heutigen Windows 8.1 ist ein wichtiger Teil einer jeden ''Operating Systems''-Lehrveranstaltung an Universitäten.
		Während fast alle Themen im Informatikstudium mit praktischen Übungen vertieft werden, wird dies meist lediglich theoretisch als historische Abhandlung vermittelt.

		In dieser Arbeit wird eine Sammlung von verschiedenen Microsoft-Betriebssystemen in Form von virtuellen Maschinen zusammengestellt.
		Mit eigens entwickelten, didaktischen Experimenten können den Studenten damit die einzelnen Entwicklungsstufen der Betriebssysteme mit ihren Innovationen näher gebracht werden.
		Für die einfache Verwendung beinhalten die virtuellen Maschinen alle notwendigen Treiber und können in die InstantLab-Plattform des Lehrstuhls für Betriebssysteme und Middleware integriert werden.

\end{abstract}
